%==============================常用宏包、环境==============================%
% !TeX program = xelatex
\documentclass[twocolumn,a4]{article}
\usepackage{xeCJK} % For Chinese characters
\usepackage{amsmath, amsthm}
\usepackage{listings,xcolor}
\usepackage{geometry} % 设置页边距
\usepackage{fontspec}
\usepackage{graphicx}
\usepackage{fancyhdr} % 自定义页眉页脚
\setsansfont{Consolas} % 设置英文字体
\setmonofont[Mapping={}]{Consolas} % 英文引号之类的正常显示,相当于设置英文字体
\geometry{left=1cm,right=1cm,top=2cm,bottom=0.5cm} % 页边距
\setlength{\columnsep}{30pt}
% \setlength\columnseprule{0.4pt} % 分割线
%==============================常用宏包、环境==============================%

%==============================页眉、页脚、代码格式设置==============================%
% 页眉、页脚设置
\pagestyle{fancy}
\lhead{\itshape The event of zero and one}
\chead{}
\rhead{Page \thepage}
\lfoot{} 
\cfoot{}
\rfoot{}
\renewcommand{\headrulewidth}{0.4pt} 
\renewcommand{\footrulewidth}{0.4pt}

% 代码格式设置
\lstset{
    language    = c++,
    %numbers     = left,
    numberstyle = \tiny,
    breaklines  = true,
    captionpos  = b,
    tabsize     = 4,
    frame       = shadowbox,
    columns     = fullflexible,
    commentstyle = \color[RGB]{0,128,0},
    keywordstyle = \color[RGB]{0,0,255},
    basicstyle   = \small\ttfamily,
    stringstyle  = \color[RGB]{148,0,209}\ttfamily,
    rulesepcolor = \color{red!20!green!20!blue!20},
    showstringspaces = false,
}
%==============================页眉、页脚、代码格式设置==============================%

%==============================标题和目录==============================%
\title{\vspace{4cm}\Huge\bfseries ACM Template}
\author{\vspace{2cm}The event of zero and one}
%\renewcommand{\today}{\number\year 年 \number\month 月 \number\day 日}
\begin{document}
\begin{titlepage}
\maketitle
\end{titlepage}

\newpage
\pagestyle{empty}
%\renewcommand{\contentsname}{目录}
\tableofcontents
\newpage\clearpage
\newpage
\pagestyle{fancy}
\setcounter{page}{1}   %new page
%==============================标题和目录==============================%

%==============================正文部分==============================%
% TODO STRING
\section{String}
\subsection{KMP}
\begin{lstlisting}
for(int i = 2, j = 0; i <= n; i ++){
    while(j && p[i] != p[j + 1]) j = ne[j];
    if(p[i] == p[j + 1]) j ++;
    ne[i] = j;
}
\end{lstlisting}

\subsection{Manacher}
\begin{lstlisting}
int manacher(string str){
	int len = str.length();
	vector<char> s(2 * len + 100);
	vector<int> d(2 * len + 100);
	int n = 2 * len + 1, res = 1;
	for(int i = 0; i < len; i ++) 
        s[2 * i + 1] = str[i];
	for(int i = 0, l = 0, r = -1; i < n; i ++) {
		int j = l + r - i;
		d[i] = max(min(d[j], j - l + 1), 0);
		if (j - d[j] < l) {
			while (i - d[i] >= 0 && i + d[i] < n && s[i - d[i]] == s[i + d[i]])
				d[i]++;
			l = i - d[i] + 1, r = i + d[i] - 1;
		}
		res = max(res, d[i]);
	}
	return res - 1;
}
\end{lstlisting}

\subsection{Trie}
\begin{lstlisting}
int son[N * 26][26], cnt[N * 26], idx;
void insert(string s){
    int p = 0;
    for(int i = 0; i < s.length(); i ++){
        int u = s[i] - 'a';
        if(!son[p][u]) son[p][u] = ++ idx;
        p = son[p][u];
    }
    cnt[p] ++;
}
int query(string s){
    int p = 0;
    for(int i = 0; i < s.length(); i ++){
        int u = s[i] - 'a'; 
        if(!son[p][u]) return 0;
        p = son[p][u];
    }
    return cnt[p];
}
\end{lstlisting}

\subsection{AC-Automation}
\begin{lstlisting}
struct AC_Automaton{
    int tr[N][26], cnt[N],fail[N], idx;
    void insert(string s){
        int p = 0;
        for (int i = 0; i < s.size(); i ++ ){
            int t = s[i] - 'a';
            if (!tr[p][t]) tr[p][t] = ++ idx;
            p = tr[p][t];
        }
        cnt[p] ++ ;
    }
    void getFail(){
        queue<int> q;
        for(int i = 0; i < 26; i ++)
            if(tr[0][i]) q.push(tr[0][i]);
        while(q.size()){
            int t = q.front(); q.pop();
            for(int i = 0; i < 26; i ++){
                int p = tr[t][i];
                if(!p) tr[t][i] = tr[fail[t]][i];
                else{
                    fail[p] = tr[fail[t]][i];
                    q.push(p);
                }
            }
        }
    }
    int query(string s){
        int res = 0;
        for (int i = 0, j = 0; i < s.size(); i ++ ){
            int t = s[i] - 'a';
            j = tr[j][t];
            for(int t = j; t && ~cnt[t]; t = fail[t])
                res += cnt[t], cnt[t] = - 1;
        }
        return res;
    }
}ac;
\end{lstlisting}

% TODO MATH
\section{Math}
\subsection{线性筛}
\begin{lstlisting}
void getPrimes(int n){
    for(int i = 2; i <= n; i ++){
        if(!st[i]) p[cnt ++] = i;
        for(int j = 0; p[j] <= n / i; j ++){
            st[p[j] * i] = 1;
            if(i % p[j] == 0) break;
        }
    }
}
\end{lstlisting}

\subsection{欧拉函数}
$\phi(n) = n\prod_{i=1}^{m}(1 - \frac{1}{p_i})$
\begin{lstlisting}
int phi(int x){
    int res = x;
    for (int i = 2; i <= x / i; i ++ ){
        if (x % i == 0){
            res = res / i * (i - 1);
            while (x % i == 0) x /= i;
        }
    }
    if (x > 1) res = res / x * (x - 1);
    return res;
}

void get_phi(int n){
    phi[1] = 1;
    for(int i = 2; i <= n; i ++){
        if(!st[i]){
            p[cnt ++] = i;
            phi[i] = i - 1;
        }
        for(int j = 0; p[j] <= n / i; j ++){
            int t = p[j] * i;
            st[t] = true;
            if(i % p[j] == 0){
                phi[t] = phi[i] * p[j];
                break;
            }
            phi[t] = phi[i] * (p[j] - 1);
        }
    }
}
\end{lstlisting}

\subsection{扩展欧几里得}
\begin{lstlisting}
int exgcd(int a, int b, int& x, int& y){
    if(!b){
        x = 1, y = 0; return a;
    }
    int d = exgcd(b, a % b, y, x);
    y -= a / b * x;
    return d; 
}
\end{lstlisting}

\subsection{逆元}
\subsubsection*{费马小定理}
若 $gcd(a, p) = 1$,且 $p$ 为质数,则有 $inv(a) \equiv a^{p - 2}(\bmod \ p)$

% TODO Structure

\section{Structure}

\subsection{DSU}
\begin{lstlisting}
struct UF {
    vector<int> fa, sz;
    int n, cnt;
    UF(int x): n(x), cnt(x), fa(x), sz(x, 1) {
        iota(fa.begin(), fa.end(), 0);
    }
    int find(int x) {
        return fa[x] == x ? x : (fa[x] = find(fa[x]));
    }
    bool merge(int x, int y) {
        x = find(x), y = find(y);
        if (x == y) return false;
        if (sz[x] < sz[y]) swap(x, y);
        fa[y] = x; sz[x] += sz[y];
        --cnt;
        return true;
    }
    bool same(int x, int y) {return find(x) == find(y);}
};
\end{lstlisting}


\subsection{Fenwick}
\begin{lstlisting}
struct Fenwick {
    const int n;
    vector<int> a;
    Fenwick(int n) : n(n), a(n) {}
    void add(int x, int v) {
        for (int i = x + 1; i <= n; i += i & -i) {
            a[i - 1] += v;
        }
    }
    int sum(int x) {
        T ans = 0;
        for (int i = x; i > 0; i -= i & -i) {
            ans += a[i - 1];
        }
        return ans;
    }
    int rangeSum(int l, int r) {
        return sum(r) - sum(l);
    }
};
\end{lstlisting}

\subsection{SegmentTree}
\begin{lstlisting}
struct SegmentTree {
	//modify1 : mul
	//modify2 : add
	struct node{
		int l, r;
		LL mul, add, sum;
	}tr[N << 2];
	void pushup(int u){
		tr[u].sum = tr[u << 1].sum + tr[u << 1 | 1].sum;
	}
	void pushdown(int u){
		auto &root = tr[u];
		auto &left = tr[u << 1];
		auto &right = tr[u << 1 | 1];
		if(root.mul != 1){
			left.mul *= root.mul;
			left.add *= root.mul;
			left.sum *= root.mul;
			right.mul *= root.mul;
			right.add *= root.mul;
			right.sum *= root.mul;
			root.mul = 1;
		}
		if(root.add){
			left.add += root.add;
			left.sum += (left.r - left.l + 1) * root.add;
			right.add += root.add;
			right.sum += (right.r - right.l + 1) * root.add;
			root.add = 0;
		}
	}
	void build(int u, int l, int r){
		tr[u] = {l, r, 1, 0, w[r]};
		if(l == r) return;
		int mid = l + r >> 1;
		build(u << 1, l, mid), build(u << 1 | 1, mid + 1, r);
		pushup(u);
	}
	void modify1(int u, int l, int r, int k){
		if(l <= tr[u].l && tr[u].r <= r){
			tr[u].mul *= k;
			tr[u].add *= k;
			tr[u].sum *= k;
			return;
		}
		else{
			pushdown(u);
			int mid = tr[u].l + tr[u].r >> 1;
			if(l <= mid) modify1(u << 1, l, r, k);
			if(r > mid) modify1(u << 1 | 1, l, r, k);
			pushup(u);
		}
	}
	void modify2(int u, int l, int r, int k){
		if(l <= tr[u].l && tr[u].r <= r){
			tr[u].add += k;
			tr[u].sum += (tr[u].r - tr[u].l + 1) * k;
			return;
		}  
		else{
			pushdown(u);
			int mid = tr[u].l + tr[u].r >> 1;
			if(l <= mid) modify2(u << 1, l, r, k);
			if(r > mid) modify2(u << 1 | 1, l, r, k);
			pushup(u);
		}
	}
	LL query(int u, int l, int r){
		if(l <= tr[u].l && tr[u].r <= r) return tr[u].sum;
		pushdown(u);
		int mid = tr[u].l + tr[u].r >> 1;
		LL res = 0;
		if(l <= mid) res += query(u << 1, l, r);
		if(r > mid) res += query(u << 1 | 1, l, r);
		return res;
	}
}t;
\end{lstlisting}

\subsection{树链剖分}
\begin{lstlisting}
const int N = 1e5 + 10, M = 2 * N;
int n, m;
int w[N], e[M], ne[M], h[N], idx;
int id[N], nw[N], cnt;
int top[N], fa[N], sz[N], son[N], dep[N];
void add(int a, int b){
    e[idx] = b, ne[idx] = h[a], h[a] = idx ++;
}
void dfs1(int u, int father, int depth){
    dep[u] = depth, fa[u] = father, sz[u] = 1;
    for(int i = h[u]; ~i; i = ne[i]){
        int j = e[i];
        if(j == father) continue;
        dfs1(j, u, depth + 1);
        sz[u] += sz[j];
        if(sz[son[u]] < sz[j]) son[u] = j;
    }
}
void dfs2(int u, int t){
    id[u] = ++ cnt, nw[cnt] = w[u], top[u] = t;
    if(!son[u]) return;
    dfs2(son[u], t);
    for(int i = h[u]; ~i; i = ne[i]){
        int j = e[i];
        if(j == fa[u] || j == son[u]) continue;
        dfs2(j, j);
    } 
}
struct node{
    int l, r; 
    int add, sum;
}tr[N * 4];
void pushup(int u){
    tr[u].sum = tr[u << 1].sum + tr[u << 1 | 1].sum;
}
void pushdown(int u){
    auto &root = tr[u];
    auto &left = tr[u << 1], &right = tr[u << 1 | 1];
    if(root.add){
        left.add += root.add;
        left.sum += (left.r - left.l + 1) * root.add;
        right.add += root.add;
        right.sum += (right.r - right.l + 1) * root.add;
        root.add = 0;
    }
}
void build(int u, int l, int r){
    tr[u] = {l, r, 0, nw[r]};
    if(l == r) return;
    int mid = l + r >> 1;
    build(u << 1, l, mid), build(u << 1 | 1, mid + 1, r);
    pushup(u);
}
void modify(int u, int l, int r, int k){
    if(l <= tr[u].l && tr[u].r <= r){
        tr[u].add += k;
        tr[u].sum += (tr[u].r - tr[u].l + 1) * k;
        return;
    }
    else{
        pushdown(u);
        int mid = tr[u].l + tr[u].r >> 1;
        if(l <= mid) modify(u << 1, l, r, k);
        if(r > mid) modify(u << 1 | 1, l, r, k);
        pushup(u);
    }
}
void modify_path(int u, int v, int k){
    while(top[u] != top[v]){
        if(dep[top[u] < dep[top[v]]]) swap(u, v);
        modify(1, id[top[u]], id[u], k);
        u = fa[top[u]];
    }
    if(dep[u] < dep[v]) swap(u, v);
    modify(1, id[v], id[u], k);
}
void modify_tree(int u, int k){
    modify(1, id[u], id[u] + sz[u] - 1, k);
}
int query(int u, int l, int r){
    if(l <= tr[u].l && tr[u].r <= r) return tr[u].sum;
    pushdown(u);
    int mid = tr[u].l + tr[u].r >> 1;
    int res = 0;
    if(l <= mid) res += query(u << 1, l, r);
    if(r > mid) res += query(u << 1 | 1, l, r);
    return res;
}
int query_path(int u, int v){
    int res = 0;
    while(top[u] != top[v]){
        if(dep[top[u]] < dep[top[v]]) swap(u, v);
        res += query(1, id[top[u]], id[u]);
        u = fa[top[u]];
    }
    if(dep[u] < dep[v]) swap(u, v);
    res += query(1, id[v], id[u]);
    return res;
}
int query_tree(int u){
    return query(1, id[u], id[u] + sz[u] - 1);
}
\end{lstlisting}

% TODO Graph
\section{Graph}

\subsection{Dijkstra}
\begin{lstlisting}
int dijkstra(){
    memset(dist, 0x3f, sizeof dist);
    priority_queue<PII,vector<PII>,greater<PII>> heap;
    heap.push({0, 1});
    while(heap.size()){
        auto t = heap.top(); heap.pop();
        int ver = t.second, distance = t.first;
        if(st[ver]) continue;
        st[ver] = 1;
        for(int i = h[ver]; i != -1; i = ne[i]){
            int j = e[i];
            if(distance + w[i] < dist[j]){
                dist[j] = distance + w[i];
                heap.push({dist[j], j});
            }
        }
    }
    if(dist[n] == 0x3f3f3f3f) return -1;
    else return dist[n];
}

\end{lstlisting}
\subsection{Spfa}
\begin{lstlisting}
bool spfa(){
    queue<int> q;
    for(int i = 1; i <= n; i ++){
        st[i] = 1, q.push(i);
    }        
    while(q.size()){
        int t = q.front(); q.pop(); st[t] = 0;
        for(int i = h[t]; i != -1; i = ne[i]){
            int j = e[i];
            if(dist[j] > dist[t] + w[i]){
                dist[j] = dist[t] + w[i];
                cnt[j] = cnt[t] + 1;
                if(cnt[j] >= n) return true;
                if(!st[j]){
                    q.push(j); st[j] = 1;
                }
            }
        }
    }
    return false;
}
\end{lstlisting}
\subsection{Prim}

\begin{lstlisting}
int prim(){
    memset(dist, 0x3f, sizeof dist);    
    int res = 0; dist[1] = 0;
    for(int i = 0; i < n; i ++){
        int t = -1;
        for(int j = 1; j <= n; j ++){
            if(!st[j] && (t == -1 || dist[t] > dist[j]) ) t = j;
        }
        if(dist[t] == 0x3f3f3f3f) return 0x3f3f3f3f;
        res += dist[t];
        st[t] = true;
        for(int j = 1; j <= n; j ++){
            dist[j] = min(dist[j], g[t][j]);
        }
    }
    return res;
}
\end{lstlisting}

\subsection{Kruskal}
\begin{lstlisting}
int kruskal(){
    sort(arr, arr + m, cmp);
    int res = 0, cnt = 1;
    for(int i = 0; i < m; i ++){
        if(find(arr[i].a) == find(arr[i].b)) continue;
        merge(arr[i].a, arr[i].b); 
        cnt ++; res += arr[i].c;
    }
    if(cnt < n) return 0x3f3f3f3f;
    else return res;
}
\end{lstlisting}

\subsection{LCA}
\begin{lstlisting}
void bfs(int root){
    memset(dep, 0x3f, sizeof dep);
    queue<int> q;
    dep[0] = 0, dep[root] = 1;
    q.push(root);
    while(q.size()){
        int t = q.front(); q.pop();
        for(int i = h[t]; ~i; i = ne[i]){
            int j = e[i];
            if(dep[j] > dep[t] + 1){
                dep[j] = dep[t] + 1; 
                fa[j][0] = t;
                q.push(j);
                for(int k = 1; k <= 15; k ++)
                    fa[j][k] = fa[fa[j][k - 1]][k - 1]; 
            }
        }
    }
}
int lca(int a, int b){
    if(dep[a] < dep[b]) swap(a, b);
    for(int k = 15; k >= 0; k --){
        if(dep[fa[a][k]] >= dep[b])
            a = fa[a][k];
    }
    if(a == b) return a;
    for(int k = 15; k >= 0; k --){
        if(fa[a][k] != fa[b][k])
            a = fa[a][k], b = fa[b][k];
    }
    return fa[a][0];
}
\end{lstlisting}

% TODO DP
\section{DP}
\subsection{LIS}
\begin{lstlisting}
int LIS(vector<int>& a){
    vector<int> stk; int n = a.size();
    stk.push_back(a[0]);
    for(int i = 1; i < n; i ++){
        if(a[i] > s1.back()) stk.push_back(a[i]);
        else *lower_bound(all(stk), a[i]) = a[i];
    }
    return stk.size();
}
\end{lstlisting}


% TODO Other
\section{Other}
\subsection{离散化}
\begin{lstlisting}
sort(all(v))
v.erase(unique(all(v)), v.end());
int find(VI& v, int& x){
    return lower_bound(all(v), x) - v.begin();
}
\end{lstlisting}

\subsection{VIM}
\begin{lstlisting}
    syntax on
    set nu
    set tabstop=4
    set shiftwidth=4
    set cin
    colo evening
    set mouse=a
\end{lstlisting}

%==============================正文部分==============================%
\end{document}